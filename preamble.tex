\usepackage[UTF8]{ctex} % latex中显示中文

\usepackage{listings} % 嵌入代码块
\usepackage{xcolor}

\usepackage{graphicx}
\usepackage{subfig}
\usepackage{float}

\usepackage{url} % \url

\usepackage{fancyhdr}

% 设置代码块的样式
\lstset{
    basicstyle          =   \sffamily,          % 基本代码风格
    keywordstyle        =   \bfseries,          % 关键字风格
    commentstyle        =   \rmfamily\itshape,  % 注释的风格,斜体
    stringstyle         =   \ttfamily,  % 字符串风格
    flexiblecolumns,                % 别问为什么,加上这个
    numbers             =   left,   % 行号的位置在左边
    showspaces          =   false,  % 是否显示空格,显示了有点乱,所以不现实了
    numberstyle         =   \zihao{-5}\ttfamily,    % 行号的样式,小五号,tt等宽字体
    showstringspaces    =   false,
    captionpos          =   t,      % 这段代码的名字所呈现的位置,t指的是top上面
    frame               =   lrtb,   % 显示边框
}
\lstdefinestyle{file}{
    basicstyle      =   \zihao{-5}\ttfamily,
    numberstyle     =   \zihao{-5}\ttfamily,
    keywordstyle    =   \color{blue},
    keywordstyle    =   [2] \color{teal},
    stringstyle     =   \color{magenta},
    commentstyle    =   \color{red}\ttfamily,
    breaklines      =   true,   % 自动换行,建议不要写太长的行
    columns         =   fixed,  % 如果不加这一句,字间距就不固定,很丑,必须加
    basewidth       =   0.5em,
}

% 定义姓名、学号
\newcommand{\me}[2]{
    \author{
      {\bfseries 姓名:}\underline{#1}\hspace{2em}
      {\bfseries 学号:}\underline{#2}\hspace{2em}\\[10pt]
  }
}

% 颜色
\newcommand{\red}[1]{
    \textcolor{red}{#1}
}

